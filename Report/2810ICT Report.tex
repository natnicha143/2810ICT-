\documentclass[12pt, a4]{report}
\usepackage[utf8]{inputenc}
\usepackage[margin=0.8in]{geometry}
\def\thesection{\arabic{section}}
\setcounter{tocdepth}{4}
\usepackage{graphicx}
\graphicspath{ {images/} }
% Below are for the code blocks
\usepackage{listings}
\usepackage{courier}
\usepackage{verbatim}
\usepackage{color}
\usepackage{rotating}

\definecolor{codegreen}{rgb}{0,0.6,0}
\definecolor{codegray}{rgb}{0.5,0.5,0.5}
\definecolor{codepurple}{rgb}{0.58,0,0.82}
\definecolor{backcolour}{rgb}{0.95,0.5,0.92}
\definecolor{bittersweet}{rgb}{1.0, 0.44, 0.37}
\definecolor{cosmiclatte}{rgb}{0.93, 0.93, 0.93}
\definecolor{eggshell}{rgb}{0.94, 0.94, 0.9}
\definecolor{fandango}{rgb}{0.71, 0.2, 0.54}

\lstdefinestyle{mystyle}{
	backgroundcolor=\color{cosmiclatte},   
	commentstyle=\color{codegreen},
	keywordstyle=\color{fandango}\small,
	numberstyle=\tiny\color{codegray},
	stringstyle=\color{codepurple},
	basicstyle=\ttfamily\footnotesize,
	breakatwhitespace=false,        
	breaklines=true,               
	captionpos=b,                    
	keepspaces=true,  
	numbers=left,                    
	numbersep=5pt,                  
	showspaces=false,                
	showstringspaces=false,
	showtabs=false,                  
	tabsize=2
}

\lstset{style=mystyle}

\title{2810, Software Technologies Assignment 1}
\author{Zaymon Foulds-Cook, s5017391 \textbar{} Natnicha Titiphanpong, s2940970}%\thanks{}}
\date{\today}

\begin{document}

\begin{titlepage}
	\maketitle
\end{titlepage}
 \tableofcontents
\pagebreak
\section{Software Technologies Assignment 1}
\subsection{Problem Statement}
	\par
	
\subsection{User Requirements}
	\textbf{The following list itemizes the user requirements for the implementation of the Ladder-gram program}
	\begin{itemize}
		\item The user should be able to interact with the program by specifying the start and goal word
		\item The user should be prompted with an option to supply a list of words that cannot be used in the path solution
		\item The user should be prompted with an option to find the shortest path from start word to goal word
	\end{itemize}
	
\subsection{Software Requirements}
	\textbf{The following list itemizes the software requirements for the implementation of the Ladder-gram program}
	\begin{itemize}
		\item The program will notify the user if the inputs are invalid
		\item The program will notify the user if a solution was found
		\item If a valid solution is found the program will print it to the console
	\end{itemize}
	\pagebreak
	
\subsection{Software Design}
	\subsubsection{High Level Design - Logical Block Diagram \textbar{} Ladder-gram}
	\begin{comment}
				\begin{figure}[h]
				\centering
				\includegraphics[scale=0.6]{Logical_Block_Laddergram}
				\caption{Logical block diagram for the Python implementation of Ladder-gram}
				\end{figure}
	\end{comment}

	\pagebreak

	\subsubsection{Structure Chart - UML \textbar{} Ladder-gram}
	\paragraph{}
	\begin{comment}
			\begin{figure}[h]
			\centering
			\includegraphics[scale=0.7]{UMLLAD}
			\caption{UML diagram for the Python implementation of Ladder-gram}
			\end{figure}
	\end{comment}


	\subsubsection{List all functions in the software}
	\textbf{	Functions in the Ladder-gram Program}
	\begin{enumerate}
		\item
			\textbf{Function name}
			\textbar{}  description
			\par Input Parameters: None
			\par Side Effects:
			\begin{itemize}
				\item side effect 1
			\end{itemize}
			\par Returns: void
	\end{enumerate}
	
	\begin{comment}
				\item
					\textbf{}
					\textbar{}  
					\par Input Parameters:
					\par Side Effects
					\begin{itemize}
					\item 
					\end{itemize}
					\par Returns:
	\end{comment}
	
	\subsubsection{List all of the data structures in the Ladder-gram program}
		\begin{enumerate}
				\item
					\textbf{Type}
					\textbar{} Purpose
					\par Used in the following functions:
					\begin{itemize}
						\item item 1
					\end{itemize}
		\end{enumerate}
	

	\subsubsection{Detailed Design}
	
	
	\textbf{Generate Solutions Pseudo-code}
	
	\begin{comment}
			\begin{figure}[h]
			\lstinputlisting[language=Python]{pseudocode_New.py}
			\caption{Psuedo-code for the generate solutions algorithm}
			\end{figure}
	\end{comment}

	
	\pagebreak
	\subsection{Configuration Management and Version Control}
	
	\subsection{Unit Tests}
		
	\subsection{Requirement Acceptance Tests}

	\subsection{User Instructions}



	
\end{document}
